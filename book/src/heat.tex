\documentclass[a4paper,draft]{book}
\usepackage[czech]{babel}
\usepackage[utf8]{inputenc}
\usepackage{tikz}
\usepackage{makeidx}
\usepackage{python}

\def\oC{{{}^\circ\mkern-2mu\rm C}}

\makeindex
\title{heating}
\author{Jiří Lidinský}
\date{Leden 2018}

\begin{document}
\frontmatter

\maketitle

\chapter*{Scope}

    Tato dokumentace popisuje řízení topení v rodinném domě manželů Lidinských
    v~Myslibořicích. V~kapitole~\ref{chap:heating} je popis topeného systému,
    popis zapojení řídícího systému je v~kapitole~\ref{chap:control}.
    Kapitola~\ref{chap:algorithm} pak obsahuje algoritmus řízení,
    popis řídícího software je uveden v kapitole xxx. Na konci v příloze jsou
    vloženy katalogové listy jak komponent topení, tak komponent řídícího
    systému.

    K~použití řídícího systému bylo přistoupeno poté, kdy bylo obytné podkroví.
    V~podkroví byla instalována krbová kamna, která jsou schopna ohřívat
    i~vodu do radiátorů.

\tableofcontents

\mainmatter

\chapter{Značení}

    V tomto dokumentu je použito hierarchické značení, které vychází z normy
    xxx. Označení každého prvku systému může být složeno z několika bloků:
    i)~umístění, ii)~funkční celek, iii)~druh předmětu, iv)~označení svorky,
    nebo přípojného místa, v)~označení signálu. Označení nemusí obsahovat
    všechny tyto komponenty, pokud nehrozí nejasnost, nebo pokud to nedává
    smysl. Každý~z bloků označení je uvozen speciálním symbolem.

\section{Umístění}

    Označení umístění, tedy označení místnosti uvnitř domu se skládá
    z~uvozovacího znaku plus a~trojciferného čísla. Přitom první číslice
    je označení podlaží a druhá číslice specifikuje místnost. Podle podlaží
    je dům členěn na přízemí a podkroví. Přízemí je označeno číslem jedna
    a~podkroví číslicí dvě. Pokud potřebuji označit něco vně domu je použita
    číslice nula. Číslování místností bylo převzato ze stavebního projektu.
    Čísla místností jsou uvedena v tabulce~\ref{tab:room-numbers}

    \input{room_numbers}

    Označení místností se používá pro prvky, nebo komponenty topení, které
    se nacházejí přímo v dané místnosti.


\section{Funkční celky}

    Označení funkčního celku je uvozeno znakem rovnítka, které je následováno
    kombinací číslic a písmen. Funkční celky se použijí pro provky nebo
    komponenty systému vytápění, které nemají přímý vztah k nějaké konkrétní
    místnosti. Použité funkční celky jsou vyjmenovány
    v~tabulce~\ref{tab:funkcni-celky}

    \begin{table}[h]
        \centering
        \begin{tabular}{r|l}
        \textsf{=KK} & krbová kamna\\
        =KKA & vnitřní okruh krbových kamen\\
        =KKB & vnější okruh krbových kamen\\
        =OT & Soustava otopných těles\\
        \textsf{=OT1} & Soustava otopných těles v přízemí\\
        =OT2 & Soustava otopných těles v podkroví\\
        =HP & Tepelné čerpadlo\\
        \end{tabular}
    \caption{Výčet funkčních celků.}
    \label{tab:funkcni-celky}
    \end{table}

\section{Druh předmětu}

    Označení druhu předmětu je uvozeno pomlčkou, která je následována
    písmenným symbolem a pořadovým číslem, je-li to nutné. Slouží k označení
    konkrétního prvku, nebo komponenty sytému vytápění. Označení druhu
    předmětu je kombinováno buďto s označením místnosti, nebo funkčního
    celku, nebo stojí samostatně.

    S~označením místnosti jsou svázány otopná tělesa. Takže, například,
    radiátor v podroví v ložnici č I. je: +201-OT. Dalším prvkem, který
    je typicky svázán s označením místnosti je čidlo měření teploty. Takže
    termostat, umístěný v přízemí v obývacím pokoji má označení: +102-TK.

    S~označením dílčího celku se pojí prvky, které nelze jednoznačně přiřadit
    ke konkrétní místnosti, ale přísluší jednoznačně k některému ze zdrojů
    topné vody, nebo k otopným tělesům přízemí nebo pokroví. Tedy, motor
    čerpadla vnitřního okruhu krbových kamen v podkroví nese označení
    =KKA-M.

    Stejné označení jako u prvků řídícího systému se nepoužívá nikde jinde,
    nehrozí proto záměna a typicky nejsou uvozeny jiným označením. Jsou to
    například vstupně výstupní moduly -B1--6, nebo relé -K1--9, jističe
    -FA1, FA2, ...

    \input{component_list}

\section{Označení svorky, přípojného místa}

    Označení svorky nebo přípojného místa je uvozeno dvojtečkou a je vždy
    použito ve spojení s druhem předměntu. Nestojí tedy nikdy samostatně.

\section{Označení signálu}

    Signál je uvozen znakem středník, následuje označení fyzikální veličiny
    a je-li to nutné pak další rozlišovací znaky. Použité fyzikální veličiny
    jsou uvedeny v tabulce~\ref(tab:fyzikalni-veliciny)

    \begin{table}[]
        \centering
        \begin{tabular}{l l l}
            $T$ & $\rm K$ & Termodynamická teplota.\\
            $t$ & $\oC$ & Teplota.\\
            $\tau$ & $\rm s$ & Čas.\\
            $Q$ & $\rm J$ & Teplo.\\
            $U$ & $\rm V$ & Napětí.\\
            $I$ & $\rm A$ & Proud.\\
            $p$ & $\rm Pa$ & Tlak.\\
            $F$ & $\rm m^3\,s^{-1}$ & Hmotnostní průtok.\\
            $m$ & $\rm kg$ & Hmotnost.\\
            $\rho$ & $\rm kg\,m^{-3}$ & Hustota.\\
            $c$ & $\rm J\,kg^{-1}\,K^{-1}$ & Měrná tepelná kapacita.\\
            $V$ & $\rm m^3$ & Objem.\\
            $P$ & $\rm W$ & Výkon, příkon.\\
        \end{tabular}
        \caption{Označení použitých fyzikálních veličin.}
        \label{tab:fyzikalni-veliciny}
    \end{table}

\section{Alternativní značení}

    Uvedené značení bohužel nekoresponduje se značením v původním projektu
    vytápění, je zde uvedena převodní tabulka mezi značením v původním
    projektu a mezi zde zavedeným značením. Bude-li někde dále v textu
    použito toto původní značení, uvedu na tomto místě upozornění.

    \begin{table}[]
        \centering
        \begin{tabular}{ l l l}
            \texttt{=KK:H;}$t$\index{$t_1$|see {=KK:H;t}} & $t_1$ & Teplota vody na výstupu z krbových kamen.
        \end{tabular}
        \caption{Výčet funkčních celků}
        \label{tab:myfirsttable}
    \end{table}

\begin{tikzpicture}
  \draw[thick] (0,0) -- (20.7,0) -- (20.7,12) -- (0,12) -- (0,0);
  \draw (0,1.5) -- (5.8,1.5) -- (5.8,0) -- (14.05,0) -- (14.05,1.5) -- (20.7,1.5);
\end{tikzpicture}

\begin{table}[]
  \centering
    \begin{tabular}{l l}
      -A1 & Napájecí rozváděč\\
      -A2 & Rozváděč s řídícím systémem\\
      -B1 & Quido RS 8/8 s montáží na lištu DIN\\
      -B2 & Quido RS 8/8 s montáží na lištu DIN\\
      -B3 & AD4ETH\\
      -B4 & DA2RS\\
      -B5 & GNOME\\
      -B6 & SENSYCON\\
      -B7 & JSP\\
      -BT & TQS3\\
      -BT & TQS3\\
      -K1 & Relé\\
      -K2 & Relé\\
      -K3 & Relé\\
      -K4 & Relé\\
      -K5 & Relé\\
      -U2 & napájecí zdroj 12V DC\\
      -U1 & napájecí zdroj 12V DC\\
      -F1 & jednofázový jistič
    \end{tabular}
  \caption{Značení komponent řídícího systému}
  \label{tab:CSLabels}
\end{table}

\chapter{Systém topení}\label{chap:heating}

    Během dostavby obytného podkroví, byla doplněna krbová kamna, HAASON+SOHN,


\chapter{Řídící systém}\label{chap:control}

    Řídící systém je realizován běžným počítačm typu PC s OS Linux, na kterém
    běží program, který v nekonečné smyčce provádí řídící alogritmus. Program
    je napsán v jazyce Java. Řídící alogoritmus je popsán pomocí textového
    souboru ve formě vzájemně propojených modulů. Tento je pak interpretován
    řídícím programem. Při změně řídícího algoritmu tak není potřeba zasahovat
    do Java kódu a poté ho znovu kompilovat. Více o konfiguraci Řídícího
    viz xxx, o řídícím programu viz xxx a konečně o algoritmu řízení viz xxx.

    Jako vstupy a výstupy do a z techlonogie jsou použity komponenty firmy
    Papouch, viz xxx. Pro binární vstupy a výstupy jsou použity moduly Quido,
    pro analogové vstupy je použit modul AD4ETH, pro analogové výstupy je
    osazen modul DA2RS. Moduly komunikují s řídícím počítačem buďto přímo
    po síti Ethernet (AD4ETH), nebo jsou do ethernetu připojeny přes převodník
    komunikační linky RS485 prostřednictvím modulu GNOME485.

    Pro měření teploty jsou osazeny dvě odporová čidla typu pt100, která jsou
    přes převodník pt100 na proudovou smyčku 4-20mA připojena k analogovým
    vstupům převodníku AD4ETH. Dále jsou využity polovodičové senzory, které
    jsou přímo připojeny k modulům Quido. Teplota v podkroví se měří pomocí
    samostatného modulu papouch xxx a venkovní teplota se měří pomocí modulu
    papouch xxx.

    Pro regulaci teploty v přízemí je použit termostat ABB xxx. Jeho kontaktní
    výstup je připojen na binární vstup modulu Quido.

    K řízení směšovacího ventilu SVxxx je použit analogový výstup modulu DA2RS.

    Solenoidové ventily jsou přes oddělovací relé připojeny k výstupům modulu
    Quido. Stejně tak čerpadla P1--3.

    Ke sběru dat z modulů xxx je použita sériová linka RS485. Ostatní komponenty
    řídícího systému jsou pak propojeny prostřednictvím komunikační sítě
    ethernet. IP adresy jsou uvedeny v tabulce xxx

    Spinel adresy modulů jsou uvedeny v tabulce xxx.

    \begin{table}
      \centering
      \begin{tabular}{ll}
        Řídící počítač & 192.168.1.90 \\
        AD4ETH         & 192.168.1.110\\
        GNOME485       & 192.168.1.111\\
      \end{tabular}
      \caption{Síťová konfigurace prvků řídícího systému}
    \end{table}

    \begin{table}
      \centering
      \begin{tabular}{ll}
        Quido1 & \\
        Quido2 & \\
        TQS3   & \\
        TQS3   & \\
        DA2    & \\
        AD4ETH &
      \end{tabular}
      \caption{Adresy spinel protokolu jednotlivých prvků}
    \end{table}

    Napájení

    \section{Teplota vody na výstupu z krbových kamen}

        Teplotní čidlo s označením =KK-BT\index{=KK-BT}, měří teplotu vody na
        výstupu z krbových kamen\index{krbová kamna}. Je umístěno na povrchu
        trubky horké smyčky, co nejblíže ke krbovým kamnům. Jedná se o odporový
        senzor typu PT100\index{PT100}. Čidlo je připojeno k převodníku
        Sensycon\index{Sensycon} s označením -B6\index{-B6}, pro převod na
        proud. Převodník má rozsah xxx. Blokové schema je na následujícím
        obrázku.

        \begin{tikzpicture}[scale=1.0]
        \end{tikzpicture}

    \section{Teplota v přízemí}

    \section{Teplota v podkroví}

    \section{Venkovní teplota}

    \section{Binární výstupy}

        Ke spínání solenoidových ventilů a čerpadel se používá modul binárních
        výstupů\index{binární výstup} Quido\index{Quido}. Výstup modulu je
        osazen relátky, s přepínacím kontaktem, který může spínat napětí do
        $60\rm\:V\:AC$, nebo do $85\rm\:V\:DC$. Maximální spínaný proud je
        $5\rm\:A$. Pro více informací viz katalogový list xxx. Aby bylo možno
        spínat spotřebiče, jejichž jmenovité napětí je $230\rm\:V\:AC$ je
        použito externí relé. Typové zapojení jednoho z binárních výstupů je
        zřejmé z obrázku~\ref{fig:binary-out}. Celkové schema zapojení je pak
        na straně xxx.

        \begin{figure}\centering
            \begin{tikzpicture}
                \input{bo}
            \end{tikzpicture}
            \caption{Typové zapojení binárních výstupů modulu Quido.}
            \label{fig:binary-out}
        \end{figure}

        \cite{da2rs}

    \chapter{Algoritmus řízení}\label{chap:algorithm}

        Z teploty kotle jsou odvozeny tři binární signály, jak je vidět z
        následujícího obrázku.

        \begin{figure}
            \begin{tikzpicture}
                \input{block}
            \end{tikzpicture}
            \caption{Vyhodnocení teploty na výstupu z krbových kamen.}
        \end{figure}

    \chapter{Trocha fyziky}\label{chap:theory}

    \chapter{Analýza naměřených dat}\label{chap:analyze}

    \chapter{Budoucí vylepšení}\label{chap:future-improvements}

    \chapter{Údržba}\label{chap:maintenance}

\appendix

    \chapter{Schemata}

    \chapter{Katalogové listy}

\backmatter

\printindex
\bibliographystyle{alpha}
\bibliography{sources}

\end{document}
