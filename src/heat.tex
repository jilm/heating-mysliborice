\documentclass{book}
\usepackage[czech]{babel}
\usepackage[utf8]{inputenc}
\usepackage{tikz}
\usepackage{makeidx}
\usepackage{python}

\makeindex
\title{heating}
\author{jlidinsky }
\date{November 2017}

\begin{document}

\maketitle

\chapter*{Scope}

    Tato dokumentace popisuje řízení topení v rodinném domě manželů Lidinských
    v Myslibořicích. V kapitole xxx je popsáno topení, popis zapojení řídícího
    systému je v kapitole xxx. Kapitola xxx pak obsahuje algoritmus řízení,
    popis řídícího software je uveden v kapitole xxx. Na konci v příloze jsou
    vloženy katalogové listy jak komponent topení, tak komponent řídícího
    systému.

\chapter{Značení}

    V tomto dokumentu je použito hierarchické značení, které vychází z normy
    xxx. Označení každého prvku systému může být složeno z několika bloků:
    i) umístění, ii) funkční celek, iii)druh předmětu, iv) označení svorky,
    nebo přípojného místa, v) označení signálu. Označení nemusí obsahovat
    všechny tyto komponenty, pokud nehrozí nedorozumění nebo pokud to nedává
    smysl. Každý z bloků označení je uvozen speciálním symbolem.

\section{Umístění}

    Označení umístění se skládá z uvozovacího znaku plus a trojciferného čísla.
    Přitom první číslice je označení podlaží a druhá číslice specifikuje
    místnost. Podle podlaží je dům členěn na přízemí a podkroví. Přízemí
    je označeno číslem jedna a podkroví číslicí dvě. Pokud potřebuji označit
    něco vně domu je použita číslice nula.

  \begin{table}[]
    \centering
    \begin{tabular}{l l}
      +1.00 & Celé přízemí\\
      +1.01 & Kuchyně\\
      +1.02 & Obývací pokoj\\
      +1.03 & Ložnice I.\\
      +1.04 & Ložnice II.\\
      +1.05 & Chodba\\
      +1.06 & Koupelna\\
      +1.07 & Záchod\\
      +1.08 & Prádelna\\
      +1.09 & Botník\\
      +1.10 & Zádveří\\
      +1.11 & Schodiště\\
      +1.12 & Garáž\\
      +1.13 & Venkovní prostor vstupních dveří\\
      +1.14 & Venkovní prostor směrem do zahrady\\
      \texttt{+200} & celé podkroví\\
      \texttt{+201} & Ložnice I.\\
      \texttt{+202} & Ložnice II.\\
      \texttt{+203} & Kuchyň a jídelna\\
      \texttt{+204} & Obývací pokoj\\
      \texttt{+205} & Koupelna\\
      \texttt{+206} & Schodiště\\
      \texttt{+207} & Technická místnost
    \end{tabular}
    \caption{Označení místností domu}
    \label{tab:rooms}
  \end{table}

\section{Funkční celky}

    Označení funkčního celku je uvozeno znakem rovnítka, které je následováno
    kombinací číslic a písmen. Funkčními celky v rámci tohoto projektu jsou:

\begin{table}[]
  \centering
    \begin{tabular}{ l l}
    \textsf{=KK} & krbová kamna\\
    =KKA & vnitřní okruh krbových kamen\\
    =KKB & vnější okruh krbových kamen\\
    =OT & Soustava otopných těles\\
    \textsf{=OT1} & Soustava otopných těles v přízemí\\
    =OT2 & Soustava otopných těles v podkroví\\
    =HP & Tepelné čerpadlo\\
    \end{tabular}
  \caption{Výčet funkčních celků}
  \label{tab:funkcnicelky}
\end{table}

\section{Druh předmětu}

    Označení druhu předmětu je uvozeno pomlčkou, která je následována
    písmenným symbolem a pořadovým číslem, je-li to nutné.

\section{Označení signálu}

    Signál je uvozen znakem středník, následuje označení fyzikální veličiny
    a je-li to nutné pak další rozlišovací znaky. Označení fyzikálních veličin
    se používají následující:

\begin{table}[]
  \centering
    \begin{tabular}{l l l}
$T$ & K & Termodynamická teplota\\
$t$ & & Teplota ve stupních celsia\\
$\tau$ & s & čas s\\
$Q$ & J & Teplo J\\
$U$ & V & Napětí V\\
$I$ & A &Proud A\\
$p$ & Pa & tlak\\
$F$ & $m^3 s^{-1}$ & průtok\\
$m$ & kg & hmotnost\\
$ro$ & & hustota\\
$c$ & & měrná tepelná kapacita\\
$V$ & & objem m3\\
    \end{tabular}
  \caption{Výčet funkčních celků}
  \label{tab:myfirsttable}
\end{table}

\section{Výjímky}

    Z důvodu přehlednosti a jednoduchosti mají následující signály a objekty
    ještě alternativní značení uvedené v tabulce

\begin{table}[]
  \centering
    \begin{tabular}{ l l l}
        \texttt{=KK:H;}$t$\index{$t_1$|see {=KK:H;t}} & $t_1$ & Teplota vody na výstupu z krbových kamen.
    \end{tabular}
  \caption{Výčet funkčních celků}
  \label{tab:myfirsttable}
\end{table}

\begin{tikzpicture}
  \draw[thick] (0,0) -- (20.7,0) -- (20.7,12) -- (0,12) -- (0,0);
  \draw (0,1.5) -- (5.8,1.5) -- (5.8,0) -- (14.05,0) -- (14.05,1.5) -- (20.7,1.5);
\end{tikzpicture}

\begin{table}[]
  \centering
    \begin{tabular}{l l}
      -A1 & Napájecí rozváděč\\
      -A2 & Rozváděč s řídícím systémem\\
      -B1 & Quido RS 8/8 s montáží na lištu DIN\\
      -B2 & Quido RS 8/8 s montáží na lištu DIN\\
      -B3 & AD4ETH\\
      -B4 & DA2RS\\
      -B5 & GNOME\\
      -B6 & SENSYCON\\
      -B7 & JSP\\
      -BT & TQS3\\
      -BT & TQS3\\
      -K1 & Relé\\
      -K2 & Relé\\
      -K3 & Relé\\
      -K4 & Relé\\
      -K5 & Relé\\
      -U2 & napájecí zdroj 12V DC\\
      -U1 & napájecí zdroj 12V DC\\
      -F1 & jednofázový jistič
    \end{tabular}
  \caption{Značení komponent řídícího systému}
  \label{tab:CSLabels}
\end{table}


\chapter{Řídící systém}

    Řídící systém je realizován běžným počítačm typu PC s OS LINUX, na kterém
    běží program, který v nekonečné smyčce provádí řídící alogritmus. Program
    je napsán v jazyce JAVA. Řídící alogoritmus je popsán pomocí textového
    souboru ve formě vzájemně propojených modulů. Tento je pak interpretován
    řídícím programem. Při změně řídícího algoritmu tak není potřeba zasahovat
    do JAVA kódu a poté ho znovu kompilovat. Více o konfiguraci Řídícího
    viz xxx, o řídícím programu viz xxx a konečně o algoritmu řízení viz xxx.

    Jako vstupy a výstupy do a z techlonogie jsou použity komponenty firmy
    Papouch, viz xxx. Pro binární vstupy a výstupy jsou použity moduly Quido,
    pro analogové vstupy je použit modul AD4ETH, pro analogové výstupy je
    osazen modul DA2RS. Moduly komunikují s řídícím počítačem buďto přímo
    po síti Ethernet (AD4ETH), nebo jsou do ethernetu připojeny přes převodník
    komunikační linky RS485 prostřednictvím modulu GNOME485.

    Pro měření teploty jsou osazeny dvě odporová čidla typu pt100, která jsou
    přes převodník pt100 na proudovou smyčku 4-20mA připojena k analogovým
    vstupům převodníku AD4ETH. Dále jsou využity polovodičové senzory, které
    jsou přímo připojeny k modulům Quido. Teplota v podkroví se měří pomocí
    samostatného modulu papouch xxx a venkovní teplota se měří pomocí modulu
    papouch xxx.

    Pro regulaci teploty v přízemí je použit termostat ABB xxx. Jeho kontaktní
    výstup je připojen na binární vstup modulu Quido.

    K řízení směšovacího ventilu SVxxx je použit analogový výstup modulu DA2RS.

    Solenoidové ventily jsou přes oddělovací relé připojeny k výstupům modulu
    Quido. Stejně tak čerpadla P1--3.

    Ke sběru dat z modulů xxx je použita sériová linka RS485. Ostatní komponenty
    řídícího systému jsou pak propojeny prostřednictvím komunikační sítě
    ethernet. IP adresy jsou uvedeny v tabulce xxx

    Spinel adresy modulů jsou uvedeny v tabulce xxx.

    \begin{table}
      \centering
      \begin{tabular}{ll}
        Řídící počítač & 192.168.1.90 \\
        AD4ETH         & 192.168.1.110\\
        GNOME485       & 192.168.1.111\\
      \end{tabular}
      \caption{Síťová konfigurace prvků řídícího systému}
    \end{table}

    \begin{table}
      \centering
      \begin{tabular}{ll}
        Quido1 & \\
        Quido2 & \\
        TQS3   & \\
        TQS3   & \\
        DA2    & \\
        AD4ETH &
      \end{tabular}
      \caption{Adresy spinel protokolu jednotlivých prvků}
    \end{table}

    Napájení

    \section{Teplota vody na výstupu z krbových kamen}

        Teplotní čidlo s označením =KK-BT\index{=KK-BT}, měří teplotu vody na výstupu
        z krbových kamen\index{krbová kamna}. Je umístěno na povrchu trubky horké smyčky,
        co nejblíže ke krbovým kamnům. Jedná se o odporový senzor typu PT100\index{PT100}.
        Čidlo je připojeno k převodníku Sensycon\index{} s označením -B6\index{-B6},
        pro převod na proud. Převodník má rozsah xxx. Blokové schema je na následujícím
        obrázku.

        \begin{tikzpicture}[scale=1.0]
        \end{tikzpicture}

    \section{Teplota v přízemí}

    \section{Teplota v podkroví}

    \section{Venkovní teplota}

    \printindex

\end{document}
